\documentclass[a4paper]{article}

\usepackage[brazil]{babel}
\usepackage[T1]{fontenc}
\usepackage[utf8]{inputenc}
\usepackage{indentfirst}

\usepackage{amsmath}
\usepackage{amssymb}

\usepackage{hyperref}

\title{Relatório Parcial de Atividades\\
  Iniciação Científica}
\author{Daniel Kiyoshi Hashimoto Vouzella de Andrade - 119025937
  \\
Polyana Tadeu -
  \\
Gustavo Muzy Fraga -
}
\date{}

\begin{document}
\maketitle

\section*{Lembretes}
\begin{itemize}
    \item Tempo médio de serviço é 1 segundo
    \item Disciplinas de fila: FCFS, LCFS
    \item Utilizações/\(\lambda\)s:
    \begin{itemize}
        \item \(0.2\)
        \item \(0.4\)
        \item \(0.6\)
        \item \(0.8\)
        \item \(0.9\)
    \end{itemize}
    \item Definir método para término da fase transiente,
        para cada utilização e disciplina
    \item IC de \(95\%\) e com precisão de \(5\%\),
        ou seja, o tamanho do intervalo deverá ser no máximo
        \(10\%\) do valor do centro do intervalo
    \item \(N = 3200\), número de rodadas
    \item Tem que usar Chi-quadrado e t-Student
    \item Temos que escolher um \(k\), número de saídas registradas,
        mais adequado (vide apostila/slides)
    \begin{itemize}
        \item Pode ser que a precisão da variância
            seja de \(5\%\),
            mas o resultado analíco não esteja dentro do intervalo
            pois \(k\) é muito pequeno e não se aproxima da normal
    \end{itemize}
    \item Métricas, para cada disciplina de fila e utilização:
    \begin{itemize}
        \item Tempo médio de espera na fila
        \item Variância do tempo de espera na fila
        \item Número médio de pessoas na fila de espera
        \item Variância do número médio de pessoas na fila de espera
    \end{itemize}
    \item Na página de rosto descreva
        a participação de cada integrante
\end{itemize}

%\newpage
\section{Introdução}
%\newpage
\section{Testes e Correção}
%\newpage
\section{Estimativa da Fase Transiente}
%\newpage
\section{Dedução dos Valores Analíticos}
%\newpage
\section{Tabelas com Resultados e Comentários Pertinentes}
%\newpage
\section{Conclusão}
%\newpage
\section{Anexo - Listagem Documentada do Programa}

\end{document}
