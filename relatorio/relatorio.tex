\documentclass[a4paper]{article}

\usepackage[brazil]{babel}
\usepackage[T1]{fontenc}
\usepackage[utf8]{inputenc}
\usepackage{indentfirst}

\usepackage{amsmath}
\usepackage{amssymb}

\usepackage{verbatim}

\usepackage{hyperref}

\newcommand{\E}[1]{E\!\left[#1\right]}

\newcommand{\arq}{\texttt}
\newcommand{\inlcode}{\texttt}
\newcommand{\lang}{\texttt}

\title{Relatório Simulador\\
  Avaliação e Desempenho - 2022.1}
\author{Daniel Kiyoshi Hashimoto Vouzella de Andrade - 119025937
  \\
Polyana Tadeu -
  \\
Gustavo Muzy Fraga -
}
\date{}

\begin{document}
\maketitle

\section*{Lembretes}
\begin{itemize}
    \item Tempo médio de serviço é 1 segundo
    \item Disciplinas de fila: FCFS, LCFS
    \item Utilizações/\(\lambda\)s:
    \begin{itemize}
        \item \(0.2\)
        \item \(0.4\)
        \item \(0.6\)
        \item \(0.8\)
        \item \(0.9\)
    \end{itemize}
    \item Definir método para término da fase transiente,
        para cada utilização e disciplina
    \item IC de \(95\%\) e com precisão de \(5\%\),
        ou seja, o tamanho do intervalo deverá ser no máximo
        \(10\%\) do valor do centro do intervalo
    \item \(N = 3200\), número de rodadas
    \item Tem que usar Chi-quadrado e t-Student
    \item Temos que escolher um \(k\), número de saídas registradas,
        mais adequado (vide apostila/slides)
    \begin{itemize}
        \item Pode ser que a precisão da variância
            seja de \(5\%\),
            mas o resultado analíco não esteja dentro do intervalo
            pois \(k\) é muito pequeno e não se aproxima da normal
    \end{itemize}
    \item Métricas, para cada disciplina de fila e utilização:
    \begin{itemize}
        \item Tempo médio de espera na fila
        \item Variância do tempo de espera na fila
        \item Número médio de pessoas na fila de espera
        \item Variância do número médio de pessoas na fila de espera
    \end{itemize}
    \item Na página de rosto descreva
        a participação de cada integrante
\end{itemize}

%\newpage
\section{Introdução}
\subsection{Funcionamento Geral do Simulador}
De forma geral,
o programa é dividido logicamente em arquivos,
onde cada um descreve uma estrutura de dados
específica para realizar uma \emph{função}.
Se for moderadamente complexa, possui alguns testes,
gerar números pseudoaleatórios e implementar uma fila,
são exemplos.

O simulador (\arq{system.c})
apenas retira e trata o evento mais próximo do \emph{instante atual},
possivelmente criando eventos com tempos pseudoaleatorios
e os colocando na lista de eventos e alterando o estado do sistema.

Para a lista de eventos (\arq{event\_heap.c}),
utilizamos uma heap de mínimo ordenada pelo tempo do eventos.
O \emph{instante atual} da simulação,
assim como o tempo de cada evento,
é armazenado em memória como um \inlcode{double}, ou seja,
um float de 64 bits.

A fila de clientes (\arq{queue.c})
é implementada usando um buffer circular extensível,
um array,
e suporta tanto a disciplina de FCFS quanto de LCFS.

A geração de números pseudoaleatórios (\arq{random.c})
é feita a partir de uma \inlcode{struct} que ``sabe''
gerar um valor de uma uniforme.
A partir disso, temos funções que usam esse valor da uniforme
para gerar valores de outras distribuições
(só foi implementado para exponencial).
Assim, criando uma espécie de \emph{interface},
tornando fácil a troca do gerador.
Excluindo nos testes, sempre usamos
um gerador baseado nas funções do \lang{C}
(\inlcode{srand} e \inlcode{rand}).

O cálculo incremental de estatísticas (\arq{stats.c}),
para saber média, variância e intervalos de confiança,
é implementado por uma estrutura que acumula amostras,
suporta tanto amostras discretas e contínuas.
As amostras contínuas são aquelas da forma
``o sistema ficou no estado X, por T tempo''.

O \emph{entry point} do simulador está na \arq{main.c},
ele usa os outros arquivos para realizar a simulação
e mostrar os resultados.
Alí está descrito como funciona a fase transiente e
coleta de informação no final das rodadas.

Também foram usados alguns arquivos extras:
\arq{args.c}, \arq{test.c}, \arq{template.c}, \arq{types.h}.
Eles têm o propósito de, respectivamente:
ler e fazer parsing dos argumentos de entrada;
ajudar a testar outras funções;
ser um arquivo base para o código; e
disponibilizar tipos com tamanho de bits já definido.
Além disso, os arquivos principais possui um sistema simples
de \inlcode{\#ifdef} e \inlcode{\#define}
que permite usar o mesmo arquivo para
\emph{header}, \emph{implementação} e \emph{teste}.
Essas informações não são muito importantes
para a compreensão do simulador,
então não serão muito comentadas pelo resto do relatório.

\subsection{Eventos}
Foram escolhidos dois tipos de evento:
\begin{itemize}
    \item \textbf{Evento de Chegada} \par
        Também é o evento inicial do sistema,
        gerado com tempo \(0\)
        pela função \inlcode{add\_first\_event}.
        Quando um evento de chegada é tratado,
        criamos um novo evento de chegada
        e inserimos na lista de eventos.
        Em seguida observamos o estado do servidor;
        se estiver ocupado, apenas
        adicionamos o cliente que chegou
        de acordo com a disciplina da fila.

        Caso o servidor esteja ocioso,
        registramos o tempo de espera
        do cliente que acabou de chegar,
        nesse caso sempre nulo,
        colocamos um evento de saída na lista de eventos,
        e marcamos o servidor como ocupado.

    \item \textbf{Evento de Saída} \par
        O evento de saída,
        é gerado quando um cliente encontra o servidor ocioso.
        Ao tratarmos o evento;
        se a fila está vazia,
        apenas indicamos que o servidor está ocioso.
        Caso contrário,
        registramos o tempo de espera na fila
        do próximo cliente a entrar em serviço
        e geramos um novo evento de saída para ele
        (inserindo esse na lista de eventos).
\end{itemize}

\subsection{Estrutura Interna}
Vamos separar as estruturas por arquivos em ordem alfabética:
\begin{itemize}
    \item \arq{event.c} \par
        Aqui descrevemos um \emph{evento} e \emph{cliente}.
        \begin{verbatim}
typedef f64 Time;
typedef u32 Color;

typedef enum _EventType {
    EVENT_arrival, EVENT_leave,
} EventType;

typedef struct _Person {
    Time arrived_time;
    Color color;
} Person;

typedef struct _Event {
    EventType type;
    Time time;
    Person person;
} Event;
        \end{verbatim}
    \item \arq{event\_heap.c} \par
        Descrevemos a \emph{lista de eventos}.
        \begin{verbatim}
typedef struct _EventHeap {
    u32 size, len;
    Event* events;
} EventHeap;
        \end{verbatim}
    \item \arq{main.c} \par
        Descrevemos as \emph{opções para a simulação},
        sendo que o último campo é
        para saber se o parsing dos argumentos foi feito com sucesso.
        \begin{verbatim}
typedef struct {
    u64 seed;
    u64 round_size;
    f64 lambda;
    f64 mu;
    u64 round_count;
    u8 lcfs;
    u8 verbose;
    u8 ARGS_valid;
} Options;
        \end{verbatim}
    \item \arq{queue.c} \par
        Descrevemos a \emph{fila de espera}.
        \begin{verbatim}
typedef enum _QueueType {
    Queue_FCFS, Queue_LCFS,
} QueueType;

typedef struct _Queue {
    QueueType type;
    u32 len;
    u32 head;
    u32 tail;
    Person *people;
} Queue;
        \end{verbatim}
    \item \arq{random.c} \par
        Descrevemos a \emph{interface
        do gerador de números pseudoaleatórios}
        e um \emph{gerador determinístico}
        (\inlcode{RandTable}).
        \begin{verbatim}
typedef enum _RandType {
    RandType_RANDC,
    RandType_Table,
} RandType;

typedef struct _RandCtx {
    f64 (*uniform)(struct _RandCtx *);
    RandType type;
} RandCtx;

typedef struct _RandTable {
    RandCtx ctx;
    f64 *table;
    u32 len;
    u32 next;
} RandTable;
        \end{verbatim}
    \item \arq{stats.c} \par
        Descrevemos o \emph{acumulador de amostras},
        \emph{intervalo de confiança} e
        os \emph{resultados da amostragem}.
        \begin{verbatim}
typedef struct _Stats  {
    u32 n;
    f64 acc;
    f64 sqr_acc;
} Stats;

typedef struct _CI {
    f64 up;
    f64 low;
    f64 precision;
} CI;

typedef struct _CachedStats {
    f64 avg;
    f64 var;
    CI tstudent;
    CI chi;
} CachedStats;
        \end{verbatim}
    \item \arq{system.c} \par
        Descrevemos o \emph{sistema} (a fila M/M/1).
        \begin{verbatim}
typedef struct _System {
    RandCtx *rand;
    f64 lambda;
    f64 mu;
    EventHeap *events;
    Queue *queue;
    b32 busy;
    Color color;
    Time curr_time;
    Stats nq_stat;
    Stats wt_stat;
} System;
        \end{verbatim}
\end{itemize}

\subsection{Linguagem Utilizada e Geração das Variáveis Aleatórias}
A linguagem utilizada foi C.
\subsection{Cores}
\subsection{Precisão IC}
\subsection{Médias e Variância}
\subsection{Máquina Utilizada e Tempo Total}

\subsection{Considerações importantes}

Devido ao uso de tipos com tamanho de bits fixo
(\inlcode{i32} e \inlcode{u64}, por exemplo)
e flags de compilação restritivas (\inlcode{-Wall -Wextra -Werror}),
pode ser que não compile por padrão em outro computador
devido aos \emph{warnings} da \inlcode{printf}.
Por exemplo, em uma arquitetura,
\inlcode{u64} vai ser um \inlcode{unsigned long}
mas em outra pode ser \inlcode{unsigned long long},
isso gera um \emph{warning} na outra arquitetura
quando a string de format contém \verb."%lu"..
Uma solução é alterar todos os
\verb."%lu". para \verb."%llu".;
outra, um pouco mais arriscada,
é desligar o \inlcode{-Werror}.

%\newpage
\section{Testes de Correção}
%\newpage
\section{Estimativa da Fase Transiente}

\newpage
\section{Dedução dos Valores Analíticos}
\subsection{Número médio de pessoas na fila}
Vamos analisar a evolução do número
deixado para trás na \(i\)-ésma e \((i+1)\)-ésima partidas.
Seja \(N_i\) e  \(N_{i+1}\) os números de pessoas
deixadas para trás nestes instantes rescpectivamente.
Seja \(K\) o número de chegadas Poisson
durante um serviço \(X\),
com \(K(z) = X^*(\lambda - \lambda \; z)\).

Para uma fila M/M/1 \(\rho = \lambda \; \E{X}\)
\begin{align*}
    K'(1) &= -\lambda \; {X^*}'(0) = \lambda \; \E{X}
        = \rho \\
    K''(1) &= \lambda^2 \; {X^*}''(0)
        = \lambda^2 \; \E{X^2} = 2 \; \rho^2 \\
    K'''(1) &= - \lambda^3 \; {X^*}'''(0)
        = \lambda^3 \; \E{X^3} = 6 \; \rho^3
\end{align*}
Desse modo:
\begin{center}
    COLOCAR IMAGEM
\end{center}
Usando tranformadas e condicionamentos, temos:
\[
    \E{z^{N_{i+1}}}
        = \E{z^{K} | N_i = 0} \; P(N_i = 0)
        + \E{z^{N_{i}+K-1} | N_i > 0} \; P(N_i > 0)
\]
No comportamento limite
\(N_i \Rightarrow N\) e \(N_{i+1} \Rightarrow N\)
com \(i \Rightarrow \infty\),
e para \(\rho < 1 \),
haverá uma distribuição estacionária
do número de pessoas na fila com T.~Z dada por
\(N(z) = \E{z^N}\).
Assim sendo, temos:
\begin{align*}
    \E{z^N} &= \E{z^K | N = 0} \; P(N = 0)
        + \E{z^{N+K-1} | N>0} \; P(N > 0) \\
    &= K(z) \; (1 - \rho)
        + \frac{K(z)}{z} \E{z^N | N > 0} \; \rho
\end{align*}
Entretando, sabemos que:
\begin{align*}
    \E{z^N} &= \E{z^N| N = 0} \; P(N = 0)
        + \E{z^N  | N > 0} \; P(N > 0) \\
    N(z) &= (1 - \rho) + \E{z^N | N > 0} \rho \\
    \E{z^N | N > 0} &= \frac{N(z) - (1 - \rho)}{\rho}
\end{align*}
Substituindo \(\E{z^N | N > 0}\)
na expressão anterior obtemos:
\begin{align*}
    N(z) &= K(z) \; (1 - \rho)
        + \frac{K(z)}{z} \; \frac{N(z) - (1 - \rho)}{\rho}
        \; \rho \\
    N(z) &= \frac{z \; K(z) \; (1 - \rho)
        + K(z) \; N(z) - K(z) \; (1 - \rho)}{z} \\
    z \; N(z) &= z \; K(z) \; (1 - \rho)
        + K(z) \; N(z) - K(z) \; (1 - \rho) \\
    N(z) \; (z - K(z)) &= K(z) \; (1 - \rho) \; (z - 1) \\
    N(z) &= \frac{K(z) \; (1 - \rho) \; (1 - z)}{K(z) - z}
\end{align*}
A partir de N(z),
podemos obter a \(\E{N}\),
uma vez que \(\E{N} = N'(1)\).
Derivando \(N(z)\) pela primeira vez:
\begin{align*}
    N(z) \; (K(z) - z) &= K(z) \; (1 - \rho) \; (1-z) \\
    N'(z) \; (K(z) - z) +  N(z) \; (K'(z) - 1)
        &= (1 - \rho) \; (K'(z) \; (1 - z) - K(z))
\end{align*}
Derivando novamente para retirar as indeterminações:
\begin{align*}
    N''(z) \; (K(z) - z) + N'(z) \; (K'(z) - 1)
        + N'(z) \; (K'(z) - 1) \\
        + N(z) \; (K''(z)) \\
    = (1 - \rho) \; (K''(z) \; (1 - z) - K(z) - K(z))
\end{align*}
\[
   N''(z) \; (K(z) - z) + 2 \; N'(z) \; (K'(z) - 1)
        + N(z) \; (K''(z))
\] \[
    \qquad\qquad \quad
    = (1 - \rho) \; (K''(z) \; (1 - z) - 2 \; K(z))
\]
Fazendo \(z = 1\):
\begin{align*}
    N''(1) \; (K(1) - 1) + 2 \; N'(1) \; (K'(1) - 1)
        + N(1) \; K''(1) \\
        = (1 - \rho) \; (K''(1) \; (1 - 1) - 2 \; K(1))
\end{align*} \begin{align*}
    2 \; N'(1) \; (\lambda \; \E{X} - 1)
        + \lambda^2 \; \E{X^2}
        &= (1 - \rho)(- 2 \; \lambda \; \E{X}) \\
    N'(1) &= \frac{(1 - \rho) \; (2 \; \lambda \; \E{X})
        + \lambda^2 \; \E{X^2}}{1 - \lambda \; \E{X}}
\end{align*}
Para M/M/1:
\[
    N'(1) = \frac{(1 - \rho) \; 2 \; \rho
        + 2 \; \rho^2}{1 - \rho} \\
        = \frac{\rho}{1 - \rho}
\]
Com \(N(z)\) calculada, é fácil achar
o número de pessoas na fila de espera,
uma vez que:
\begin{center}
    COLOCAR IMAGEM
\end{center}
\begin{align*}
    \E{z^{N_q}} &= \E{z^N| N = 0} \; P(N = 0)
        + \E{z^{N-1} | N > 0} \;P(N > 0) \\
    N_q(z) &= (1 - \rho) + \frac{N(z) - (1 - \rho)}{z \; \rho}
        \; \rho \\
    N_q(z) &= \frac{z \; (1 - \rho) + N(z) - (1 - \rho)}{z} \\
    N_q(z) &= \frac{N(z) - (1 - \rho) \; (1 - z)}{z} \\
\end{align*}
Derivando \(N_q(z)\) uma vez para calcular o primeiro momento
\(\E{N_q} = N_q'(1)\):
\begin{align*}
    N_q(z) \; z &= N(z) - (1 - \rho) \; (1 - z) \\
    N_q'(z) \; z + N_q(z) &= N'(z) + (1 - \rho)
\end{align*}
Fazendo \(z = 1\):
\begin{align*}
    N_q'(1) + N_q(1) &= N'(1) + (1 - \rho) \\
    N_q'(1) &= N'(1) - \rho \\
    &= \frac{\rho}{1 - \rho} - \rho \\
    &= \frac{\rho^2}{1 - \rho}
\end{align*}

\subsection{Variância do número de pessoas na fila}
Para achar a variância do número de pessoas na fila,
precisamos encontrar o segundo momento de \(N_q\),
que pode ser dados por \(E[N_q^2] = N_q'(1) + N_q''(1)\),
então precisamos derivar \(N_q(z)\) novamente.
\[
    N_q''(z) \; z + N_q'(z) + N_q'(z) = N''(z)
\]
Fazendo \(z = 1\):
\begin{align*}
    N_q''(1) \; 1 + N_q'(1) \; + N_q'(1) &= N''(1) \\
    N_q''(1) &= N''(1) - 2 \; N_q'(1)
\end{align*}
Logo,
\begin{align*}
    \E{N_q^2} &= N'(1) - \rho + N''(1) - 2 \; N_q'(1) \\
    \E{N_q^2} &= N'(1) - \rho + N''(1) - 2 \; (N'(1) - \rho) \\
    \E{N_q^2} &= \E{N^2} - 2 \; \E{N} + \rho
\end{align*}
Como já temos o primeiro momento \(\E{N}\) calculado,
precisamos agora achar o segundo momento \(\E{N^2}\).
Derivando \(N(z)\) pela terceira vez vamos obter:
\begin{align*}
    N'''(z) \; (K(z) - z) + N''(z) \; (K'(z) - 1)
        + 2 \; N''(z) \; (K'(z) - 1) \\
        + 2 \; N'(z) \; (K''(z)) + N'(z) \; (K''(z))
        + N(z) \; (K'''(z)) \\
        = (1 - \rho) \; (-2 \; K''(z) - K''(z)
        + (1 - z) \; K'''(z))
\end{align*} \begin{align*}
    N'''(z) \; (K(z) - z) + 3 \; N''(z) \; (K'(z) - 1)
        + 3 \; N'(z) \; K''(z) \\
        + N(z) \; K'''(z) \\
        = (1 - \rho) \; (-3 \; K''(z) + (1 - z) \; K'''(z))
\end{align*}
Fazendo \(z = 1\):
\begin{align*}
    &3 \; N''(1) \; (K'(1) - 1)
        + 3 \; N'(1) \; K''(1) + K'''(1)
        = (1 - \rho) \; (-3 \; K''(1)) \\
    &3 \; N''(1) \; (1 - \lambda \; \E{X})
        - 3 \; N'(1) \; (\lambda^2 \; \E{X^2})
        - \lambda^3 \; \E{X^3}
        = (1 - \rho) \; (3 \; \lambda^2 \; \E{X^2}) \\
    &3 \; N''(1) \; (1 - \lambda \; \E{X})
        = (1 - \rho) \; (3 \; \lambda^2 \; \E{X^2})
        + 3 \; N'(1) \; (\lambda^2 \; \E{X^2})
        + \lambda^3 \; \E{X^3} \\
    &3 \; N''(1) \; (1 - \lambda \; \E{X})
        = (1 - \rho) \; (3 \; \lambda^2 \; \E[X^2])
        + \frac{3 \; (\lambda^2 \; \E{X^2}) \;
        ((1 - \rho) \; (2 \; \lambda \; \E{X})
        + \lambda^2 \; \E{X^2})}{2 \; (1 - \lambda \; \E{X})}\\
        &+ \lambda^3 \; \E{X^3} \\
\end{align*}
\begin{align*}
    &3\;N^{''}(1)\;(1-\lambda E[X]) = \\
    &\frac{6 \; (1 - \rho) \; (\lambda^2 \; \E{X^2}) \;
        (1 - \lambda \; \E{X}) + 3 \; (\lambda^2 \; \E{X^2})
        \; ((1 - \rho) \; (2 \; \lambda \; \E{X})
        + \lambda^2 \; \E{X^2})}{2 \; (1 - \lambda \; \E{X})}\\
        &+\frac{2 \; (1 - \lambda \; \E{X})
        \;(\lambda^3 \; \E{X^3})}{2 \; (1 - \lambda \; \E{X})}
\end{align*}
\begin{align*}
    &N''(1) = \\
    &\frac{6 \; (1 - \rho) \; (\lambda^2 \; \E{X^2})
        \; (1 - \lambda \; \E{X})
        + 3 \; (\lambda^2 \; \E{X^2}) \; ((1 - \rho)
        \; (2 \; \lambda \; \E[X]) + \lambda^2 \; \E{X^2})
        }{6 \; (1 - \lambda \; \E{X})^2}\\
        &+\frac{2 \; (1 - \lambda \; \E{X}) \; \lambda^3
        \; \E[X^3]}{6 \; (1 - \lambda \; \E{X})^2}
\end{align*}
Logo \(\E{N^2} = N''(1) + N'(1)\)
para uma fila M/M/1, pode ser dado por:
\begin{align*}
    &\frac{12 \; (1 - \rho)^2 \; \rho^2
        + 6 \; \rho^2 \; ((1 - \rho) \; 2 \; \rho
        + 2 \; \rho^2) + 12 \; (1 - \rho) \; \rho^3
        + 3 \; (1 - \rho) \; (2 \; \rho (1 - \rho)
        + 2 \; \rho^2)}{6 \; (1 - \rho)^2} \\
    &\frac{6 \; \rho - 6 \; \rho^2 + 12 \; \rho^2
        - 24 \; \rho^3 + 12 \; \rho^4 + 12 \; \rho^3
        + 12 \; \rho^3 - 12 \; \rho^4}{6 \; (1 - \rho)^2} \\
    &\frac{6 \; \rho^2 + 6 \; \rho}{6 \; (1 - \rho)^2} \\
    &\frac{\rho^2 + \rho}{(1 - \rho)^2}
\end{align*}
Agora podemos calcular \(E[N_q^{2}]\)
\begin{align*}
    \E{N_q^2} &= \frac{\rho^2 + \rho}{(1 - \rho)^2}
        - \frac{2 \; \rho}{(1 - \rho)} + \rho \\
    &= \frac{\rho^2 + \rho - 2 \;\rho + 2 \; \rho^2
        + \rho - 2 \; \rho^2 + \rho^3}{(1 - \rho)^2} \\
    &= \frac{\rho^2 + \rho^3}{(1 - \rho)^2}
\end{align*}
Por fim, calculamos a variância do número de pessoas
na fila de espera de uma M/M/1:
\begin{align*}
    V(N_q) &= \frac{\rho^2 + \rho^3}{(1 - \rho)^2}
        - (\frac{\rho^2}{1 - \rho})^2 \\
    &=\frac{\rho^2 + \rho^3 - \rho^4}{(1-\rho)^2}
\end{align*}

\subsection{Tempo médio em uma fila de espera FCFS}

Acompanhando um freguês tíıpico em uma fila M/G/1 FCFS,
podemos observar que o número deixado para trás é
exatamente o número de pessoas que chegou
durante o tempo \(T\) que o freguês típico passou no sistema.
Como o processo de chegada é Poisson
e relacionando as chegadas ao intervalo de tempo em
que as chegadas ocorreram,
pode-se escrever
\[
    N(z) = T(\lambda - \lambda \; z)
\]
Fazendo \(s = \lambda - \lambda \; z\)
e substituindo \(z = \frac{\lambda - s}{\lambda}\),
temos:
\[
    T^*(s) =
    \frac{(1 - \rho)\; s \; X^*(s)}{s - \lambda + \lambda \; X^*(s)}
\]
Sabemos que \(T = W + X\), como \(W\) e \(X\)
são independentes em uma fila FCFS,
então  \(T^*(s) =  W^*(s) \; X^*(s)\),
logo a transformada de Laplace
da pdf do tempo de espera na fila é dada por:
\[
    W^*(s) = \frac{(1 - \rho) \; s}{s - \lambda + \lambda \; X^*(s)}
\]
O tempo médio de espera na fila
pode ser facilmente obtido derivando \(W^{*}(s)\),
uma vez que \(E[W] = - W^{*'}(0)\).
Dessa forma, derivando em relação a \(s\) temos:
\begin{align*}
    W^*(s)(s - \lambda + \lambda \; X^*(s)) &= (1 - \rho) \; s \\
    {W^*}'(s) \; (s - \lambda + \lambda \; X^*(s))
        + W^*(s) \; (1 + \lambda \; {X^*}'(s)) &= (1 - \rho)
\end{align*}
Derivando novamente para retirar a ideterminação:
\begin{align*}
    {W^*}''(s) \; (s - \lambda + \lambda \; X^*(s))
        + {W^*}'(s) \; (1 + \lambda \; {X^*}'(s)) \\
        + {W^*}'(s) \; (1 + \lambda \; {X^*}'(s))
        + W^*(s) \; (\lambda \; {X^*}''(s)) = 0
\end{align*}
Fazendo \(s = 0\)
\begin{align*}
    2 \; {W^*}'(0) \; (1 + \lambda \; {X^*}'(0))
        &= - \lambda \; {X^*}''(0) \\
    {W^*}'(0)
        &= \frac{-\lambda \; {X^*}''(0)}{1 + \lambda \; {X^*}'(0)} \\
    \E{W} &= \frac{\lambda \; \E{X^2}}{2 \; (1 - \lambda \; \E{X})}
\end{align*}
Como \(\E{X^2} = 2 \; \E{X} \; \E{Xr}\),
podemos escrever
\(
    \E{W} = \frac{\lambda \; 2 \; \E{X} \; \E{Xr}}{2
        \; (1 - \lambda \; \E{X})}
\).
Para a fila M/M/1,
como temos um serviço exponencial,
\(E[Xr] = E[X]\),
logo \(\E{W} = \frac{\rho \; \E{X}}{(1 - \rho)}\).
Para o nosso simulador o tempo médio de serviço é 1 segundo,
então
\[
    \E{W} = \frac{\rho}{(1 - \rho)}
\]

\subsection{Variância do tempo de espera em uma fila de espera FCFS}
Para o cálculo da variância
é preciso achar o segundo momento \(\E{W^2} = {W^*}''(0)\).
Assim sendo, derivando \(W^*(s)\) pela terceira vez:
\begin{align*}
    {W^*}''(s) \; (s - \lambda + \lambda \; X^*(s))
        + {W^*}'(s) \; (1 + \lambda \; {X^*}'(s))
        + {W^*}'(s) \; (1 + \lambda \; {X^*}'(s))
        + W^*(s)(\lambda \; {X^*}''(s)) = 0 \\
    {W^*}'''(s) \; (s - \lambda + \lambda \; X^*(s))
        + {W^*}''(s) \; (1 + \lambda \; {X^*}'(s))
        + {W^*}''(s) \; (1 + \lambda \; {X^*}'(s))
        + {W^*}'(s) \; (\lambda \; {X^*}''(s)) \\
        + {W^*}''(s) \; (1 + \lambda \; {X^*}'(s))
        + {W^*}'(s) \; (\lambda \; {X^*}''(s))
        + {W^*}'(s) \; (\lambda \; {X^*}''(s))
        + W^*(s) \; (\lambda \; {X^*}'''(s)) = 0
    \end{align*}
Fazendo \(s = 0\):
\[
    3 \; {W^*}''(0) \; (1 + \lambda {X^*}''(s))
        + 3 {W^*}'(0) \; (\lambda \; {X^*}''(0))
        + \lambda \; {X^*}'''(0) = 0
\]
\begin{align*}
    3 \; {W^*}''(0) \; (1 - \lambda \; {X^*}''(s))
        &= -3 \; {W^*}'(0) \; (\lambda \; {X^*}''(0))
        - \lambda {X^*}'''(0) \\
    3 \; {W^*}''(0) \; (1 - \lambda \E{X^2})
        &= 3 \; \E{W} \; (\lambda \; \E{X^2})
        + \lambda \; \E{X^3} \\
    {W^*}''(0) &= \frac{3 \; \lambda \; \E{X^2}
        \; \lambda \; \E{X^2}}{6 \; (1 - \lambda \; \E{X^2})^2}
        + \frac{\lambda \; \E{X^3}}{3
        \; (1 - \lambda \; \E{X^2})} \\
    {W^*}''(0) &= \frac{\lambda \; \E{X^2} \; \lambda
        \; \E{X^2}}{2 \; (1 - \lambda \; \E{X^2})^2}
        + \frac{\lambda \; \E{X^3}}{3
        \; (1 - \lambda \; \E{X^2})} \\
    {W^*}''(0) &= 2 \; \E{W}^2
        + \frac{\lambda \; \E{X^3}}{3
        \; (1 - \lambda \; \E{X^2})} \\
    {W^*}''(0) &= E[W^2]
\end{align*}
Logo, para a fila M/M/1
\[
    \E{W^2}
    = 2 \; \E{W}^2 + \frac{\lambda \; \E{X^3}}{3 \; (1 - \rho)}
\]
Finalmente, podemos achar a variância para uma fila M/M/1 FCFS:
\begin{align*}
    V(W_{FCFS}) &= \E{W^2} - \E{W}^2 \\
    &= 2 \; \E{W}^2 + \frac{\lambda \; \E{X^3}}{3 \; (1 - \rho)}
        - \E{W}^2 \\
    &= \E{W}^2 + \frac{\lambda \; \E{X^3}}{3 \; (1 - \rho)} \\
    &= \frac{\lambda^2 \; \E{X^2}^2}{4 \; (1 - \rho)^2}
        + \frac{\lambda \; \E{X^3}}{3 \; (1 - \rho)} \\
    &= \frac{3 \; \lambda^2 \; \E{X^2}^2 + 4 \; \lambda \; \E{X^3}
        - 4 \; \rho \; \lambda \; \E{X^3}}{12 \; (1 - \rho)^2}
    \end{align*}
Como o tempo médio de serviço é 1 segundo, \(\lambda = \rho\)
\[
    V(W_{FCFS}) = \frac{3 \; \rho^2 \; \E{X^2}^2
        + 4 \; \rho \; \E{X^3} - 4 \; \rho^2 \; \E{X^3}}{12
        \; (1 - \rho)^2}
\]
O segundo e o terceiro momentos de uma exponencial,
com \(\mu = 1\) podem ser dados por:
% TODO: BOTAR INTEGRAL
\begin{align*}
    &\E{X^2} \int = \frac{2}{\mu^2 } = 2 \\
    &\E{X^3} \int = \frac{6}{\mu^3 } = 6
\end{align*}
Então,
\[
    V(W_{FCFS})
    = \frac{-12 \; \rho^2 + 24 \; \rho}{12 \; (1 - \rho)^2} = \frac{-\rho^2 + 2 \; \rho}{(1 - \rho)^2}
\]
\subsection{Tempo médio em uma fila de espera LCFS}
Quando um freguêes típico chega à fila, o instante de chegada
é equivalente a uma amostragem aleatória no tempo (PASTA)
e a fila  é encontrada com seus valores médios no tempo,
tanto na fila de espera quanto no serviço.
O serviço médio pendente neste instante  é dado
por \(\E{U_0} = \E{N_q}\;\E{X} + \E{N_s}\;\E{Xr}\), com \(\E{N_q} = \lambda\;\E{W}\).

Deste serviço pendente, a parte \(\E{W_0} = \E{N_s}\;\E{Xr} = \rho\;\E{Xr}\)
ser;a executada antes da entrada do freguês típico em serviço.
Caso novas chegadas nãpo ocorressem após a chegada
do freguês típico, o tempo médio de espera seria dado simplesmente
pelo atraso \(\E{W_0}\). A expressão para \(\E{W_0}\) pode ser também
obtida condicionando no estado ocupado ou ocioso
encontrado ao chegar:
\[
    \E{W_0} = \E{W_0|ocupado} \; P(ocupado) + \E{W_0|ociso} \; P(ocioso) = \rho\;\E{Xr}
\]
Se a fila  é encontrada ociosa o atraso até entrada em serviço  é zero e se o servidor
está ocupado o serviço residual  é o atraso mínimo a ser observado.
Entretanto, toda nova chegada que ocorre enquanto o fregues típico
ainda está na fila deespera será servida antes do freguês típico.
O tempo médio de espera na fila pode ser visto
como um período ocupado diferenciado iniciado por \(\E{W_0}\), logo:
\[
    \E{W_{LCFS}} = \frac{ \E{W_0}}{ 1-\rho} = \frac{ \rho\;\E{Xr}}{ 1-\rho},\;\rho < 1
\]
Como na fila M/M/1 \(\E{Xr} = \E{X}\), pela propriedade da falta de memória, e  \(\E{X} =1s\)
\(\E{W_LCFS}\) pode ser escrito como \(\E{W_{LCFS}} =  \frac{ \rho}{ 1-\rho}\), que
é o mesmo tempo médio de espera em uma fila FCFS, como esperado e discutido em aula.
\subsection{Variância do tempo de espera em uma fila LCFS}
Para calcular a variância do tempo de espera em uma fila LCFS, precisamos obter
a sua transfromada \(W_{LCFS}^*(s)\), que pode ser obtida diretamente da T.L
do período ocupado, condicionando no estado ocupado ou ocioso que a fila  é encontrada pelo freguês tépico.
Quando o sistema é encontrado ocupado, o serviço inicial do período ocupado  é dado pela vida residual do
serviço, que no caso da fila M/M/1 é o proprio serviço.
\begin{align*}
    W_{LCFS}^*(s) &= \E{e^{s\;W_{LCFS}}|ocupado} \; P(ocupado) + \E{e^{s\;W_{LCFS}}|ocioso} \; P(ocioso)\\
    &= \rho\;  X^*(s+\lambda- \lambda\;G^*(s)) + (1-\rho)\\
    &=  \rho\; G^*(s) + (1-\rho)
\end{align*}
Derivando uma vez e fazendo s = 0 obetmos o primeiro momento:
\[
{W_{LCFS}^*}'(0) =  \rho\; {G^*}'(0) = \frac{ \rho\;\E{X}}{ 1-\rho}
\]
Como esperado e visto anteriormente. Agora derivamos novamente, para achar o segundo momento:
\[
{W_{LCFS}^*}''(0) =  \rho\; {G^*}''(0) = \frac{ \rho\;\E{X^2}}{(1-\rho)^3}
\]
Para achar a variância :
\begin{align*}
V(W_{LCFS})&= \frac{ \rho\;\E{X^2}}{(1-\rho)^3} - (\frac{ \rho\;\E{X}}{ 1-\rho})^2\\
&= \frac{ 2\; \rho}{(1-\rho)^3} - (\frac{ \rho}{ 1-\rho})^2\\
&=\frac{ 2\; \rho -\rho^2 + \rho^3}{(1-\rho)^3}
\end{align*}

Ao final obtivemos os seguintes valores analíticos:

\begin{table}[h]
\centering
\caption{Valores analíticos}
\vspace{0.9cm}
\begin{tabular}{r|lr}

Variável & Média & Variância \\ % Note a separação de col. e a quebra de linhas
\hline                               % para uma linha horizontal
\(W_{LCFS}\) & \(\frac{ \rho}{ 1-\rho}\)  & \(\frac{ 2\; \rho -\rho^2 + \rho^3}{(1-\rho)^3}\) \\
\(W_{FCFS}\) & \(\frac{ \rho}{ 1-\rho}\)  & \(\frac{2 \; \rho -\rho^2 }{(1 - \rho)^2}\) \\
\(N_q\)      & \(\frac{ \rho^2}{ 1-\rho}\)& \(\frac{\rho^2 + \rho^3 - \rho^4}{(1 - \rho)^2}\) \\

\end{tabular}
\end{table}
\newpage
\section{Tabelas com Resultados e Comentários Pertinentes}
%\newpage
\section{Conclusão}
%\newpage
\section{Anexo - Listagem Documentada do Programa}

\end{document}
