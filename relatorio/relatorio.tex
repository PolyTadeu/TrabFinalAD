\documentclass[a4paper]{article}

\usepackage[brazil]{babel}
\usepackage[T1]{fontenc}
\usepackage[utf8]{inputenc}
\usepackage{indentfirst}

\usepackage{amsmath}
\usepackage{amssymb}

\usepackage{hyperref}

\title{Relatório Parcial de Atividades\\
  Iniciação Científica}
\author{Daniel Kiyoshi Hashimoto Vouzella de Andrade - 119025937
  \\
Polyana Tadeu -
  \\
Gustavo Muzy Fraga -
}
\date{}

\begin{document}
\maketitle

\section*{Lembretes}
\begin{itemize}
    \item Tempo médio de serviço é 1 segundo
    \item Disciplinas de fila: FCFS, LCFS
    \item Utilizações/\(\lambda\)s:
    \begin{itemize}
        \item \(0.2\)
        \item \(0.4\)
        \item \(0.6\)
        \item \(0.8\)
        \item \(0.9\)
    \end{itemize}
    \item Definir método para término da fase transiente,
        para cada utilização e disciplina
    \item IC de \(95\%\) e com precisão de \(5\%\),
        ou seja, o tamanho do intervalo deverá ser no máximo
        \(10\%\) do valor do centro do intervalo
    \item \(N = 3200\), número de rodadas
    \item Tem que usar Chi-quadrado e t-Student
    \item Temos que escolher um \(k\), número de saídas registradas,
        mais adequado (vide apostila/slides)
    \begin{itemize}
        \item Pode ser que a precisão da variância
            seja de \(5\%\),
            mas o resultado analíco não esteja dentro do intervalo
            pois \(k\) é muito pequeno e não se aproxima da normal
    \end{itemize}
    \item Métricas, para cada disciplina de fila e utilização:
    \begin{itemize}
        \item Tempo médio de espera na fila
        \item Variância do tempo de espera na fila
        \item Número médio de pessoas na fila de espera
        \item Variância do número médio de pessoas na fila de espera
    \end{itemize}
    \item Na página de rosto descreva
        a participação de cada integrante
\end{itemize}

%\newpage
\section{Introdução}
%\newpage
\section{Testes e Correção}
%\newpage
\section{Estimativa da Fase Transiente}
%\newpage
\section{Dedução dos Valores Analíticos}
\begin{itemize}
    \item Número médio de pessoas na fila\\\\
    Vamos analisar a evolução do número deixado para trás na i-ésma e i+1-ésima partidas. Seja \(N_i\) e  \(N_{i+1}\) os núemro de pessoas deixadas para trás nestes instantes rescpectivamente. Seja \(K\) o número de chegadas POisson durante um serviço \(X\), com \(K(z) = X^*(\lambda - \lambda\;z)\). Para uma fila M/M/1 \(\rho = \lambda\;E[X]\)
    \begin{center}
         \begin{align*}
            &K^{'}(1) = -\lambda\;X^{*'}(0) = \lambda\; E[X] = \rho\\
            &K^{''}(1) = \lambda^2\;X^{*''}(0) = \lambda^2\; E[X^2] = 2\;\rho^2\\
            &K^{'''}(1) = -\lambda^3\;X^{*'''}(0) = \lambda^3\; E[X^3] = 6\;\rho^3
         \end{align*}
    \end{center}
    Desse modo:
    \begin{center}
        COLOCAR IMAGEM
    \end{center}
    Usando tranformadas e condicionamentos, temos: 
    \begin{center}
        \(E[z^{N_{1}+1}] = E[z^{K}| N_i = 0]P(N_i = 0) + E[z^{N_{i}+K-1} | N_i>0]P(N_i>0)\)
    \end{center}
    No comportamento limite \(N_i \Rightarrow N\) e \(N_{i+1} \Rightarrow N\) com \(i \Rightarrow \infty\), e para \(\rho < 1 \), haverá uma distribuição estacionário do número de pessoas na fila com T.Z dada por \(N(z) = E[z^N]\). Assim sendo, temos:
     \begin{center}
         \begin{align*}
            E[z^N] &= E[z^{K}| N = 0]P(N = 0) + E[z^{N+K-1} | N>0]P(N>0)\\
            &=K(z)(1-\rho)+ \frac{K(z)}{z} E[z^{N} | N>0]\rho
         \end{align*}
    \end{center}
    Entretando, sabemos que: 
    \begin{center}
         \begin{align*}
            &E[z^N] = E[z^{N}| N = 0]P(N = 0) + E[z^{N} | N>0]P(N>0)\\
            &E[z^N] =(1-\rho)+ E[z^{N} | N>0]\rho\\
            &E[z^{N} | N>0] = \frac{N(z)-(1-\rho)}{\rho}
         \end{align*}
    \end{center}
    Substituindo \(E[z^{N} | N>0]\) na expressão anterior obtemos:
    \begin{center}
         \begin{align*}
            N(z) &= K(z)\;(1-\rho) + \frac{K(z)}{z}\; \frac{N(z)-(1-\rho)}{\rho}\; \rho\\
            N(z) &= \frac{z\;K(z)\;(1-\rho) + k(z)\;N(z) - K(z)\;(1-\rho)}{z}\\
            z\;N(z) &= z\;K(z)\;(1-\rho) + k(z)\;N(z) - K(z)\;(1-\rho)\\
            N(z)\;(z-k(z)) &= \;K(z)\;(1-\rho)\; (z-1)\\
            N(z) &= \frac{K(z)\;(1-\rho)\; (1-z)}{K(z)-z} 
         \end{align*}
    \end{center}
    Apartir de N(z), podemos obter a E[N], uma vez que \(E[N] = N^{'}(1)\). Derivando N(z) pela primeira vez:
    \begin{center}
         \begin{align*}
           N(z)\;(K(z)-z) &= K(z)\;(1-\rho)\; (1-z)\\
           N^{'}(z)\;(K(z)-z) +  N(z)\;(K^{'}(z)-1) &= (1-\rho)(K^{'}(z)\;(1-z) - K^(z))\\ 
         \end{align*}
    \end{center}
    Derivando novamente para retirar as indeterminações:
    \begin{align*}
       N^{''}(z)\;(K(z)-z) + N^{'}(z)\;(K^{'}(z)-1) +  N^{'}(z)\;(K^{'}(z)-1) +  N(z)\;(K^{''}(z)) =\\ (1-\rho)(K^{''}(z)\;(1-z) - K(z)- K(z))\\ 
    \end{align*}
    Fazendo z = 1:
    \begin{align*}
       N^{''}(1)\;(K(1)-1) +2 N^{'}(1)\;(K^{'}(1)-1) +  N(1)\;(K^{''}(1)) &= (1-\rho)(K^{''}(z\1)\;(1-1) - 2K(1))\\ 
       2 N^{'}(1)\;(\lambda\;E[X]-1) + \lambda^2\; E[X^2] \; &= (1-\rho)(- 2\lambda\;E[X])\\ 
       N^{'}(1) &= \frac{(1-\rho)(2\lambda\;E[X])+ \lambda^2\; E[X^2]}{1-\lambda E[X]}
    \end{align*}
    Para M/M/1
    \begin{align*}
      N^{'}(1) = \frac{(1-\rho)\;2\;\rho+ 2\;\rho^2 }{1-\rho} = \frac{\rho}{1-\rho}
    \end{align*}
    Com N(z) calculada, é fácil achar o número de pessoas na fila de espera, uma vez que:
    \begin{center}
        COLOCAR IMAGEM
    \end{center}
    \begin{center}
         \begin{align*}
            E[z^{Nq}] &= E[z^{N}| N = 0]P(N = 0) + E[z^{N-1} | N>0]P(N>0)\\
            Nq(z)&=(1-\rho)+ \frac{N(z) -  (1-\rho)}{z\;\rho} \;\rho\\
            Nq(z)&=\frac{z\;(1-\rho) + N(z) -  (1-\rho)}{z} \;\\
            Nq(z)&=\frac{N(z) -  (1-\rho)(1-z)}{z} \;\\
         \end{align*}
    \end{center}
    Derivando Nq(z) uma vez para calcular o primeiro momento \(E[Nq] = Nq^{'}(1)\).
    \begin{center}
         \begin{align*}
            Nq(z)\;z&=N(z) -  (1-\rho)(1-z)\\
            Nq^{'}(z)\;z + Nq(z)&=N^{'}(z) + (1-\rho)
         \end{align*}
    \end{center}
    Fazendo z = 1 
    \begin{center}
         \begin{align*}
            Nq^{'}(1) + Nq(1)&=N^{'}(1) + (1-\rho)\\
            Nq^{'}(1) &=N^{'}(1) -\rho \;=\; \frac{\rho}{1-\rho} -\rho \;=\;  \frac{\rho^2}{1-\rho}
         \end{align*}
    \end{center}
\end{itemize}
\begin{itemize}
    \item Variância do número de pessoas na fila\\\\
    Para achar a variância do núemro de pessoas na fila, precisamos encontrar o segundo momento de Nq, que pode ser dados por \(E[Nq^2] = Nq^{'}(1) + Nq^{''}(1)\), então precisamos derivar Nq(z) novamente.
    \begin{center}
            \(Nq^{''}(z)\;z + Nq^{'}(z)\; + Nq^{'}(z)=N^{''}(z)\)
    \end{center}
    Fazendo z = 1 
    \begin{center}
         \begin{align*}
            Nq^{''}(1)\;1 + Nq^{'}(1)\; + Nq^{'}(1)&=N^{''}(1)\\
            Nq^{''}(1) &=N^{''}(1) - 2\;Nq^{'}(1)
         \end{align*}
    \end{center}
    Logo, 
    \begin{center}
         \begin{align*}
            E[Nq^2] &= N^{'}(1) -\rho + N^{''}(1) - 2\;Nq^{'}(1)\\
            E[Nq^2] &= N^{'}(1) -\rho + N^{''}(1) - 2\;(N^{'}(1) -\rho)\\
            E[Nq^2] &= E[N^2]-2\;E[N] + \rho
         \end{align*}
    \end{center}
    Como já temos o primeiro momento \(E[N]\) calculado, precisamos agora achar o segundo momento \(E[N^2]\). Derivando \(N(z)\) pela terceura vez vamos obter:
    \begin{align*}
           N^{'''}(z)\;(K(z)-z) +N^{''}(z)\;(K^{'}((z)-1)+ 2N^{''}(z)\;(K^{'}(z)-1) +\\
           2\;N^{'}(z)\;(K^{''}(z)) + N^{'}(z)\;(K^{''}(z))+  N(z)\;(K^{'''}(z)) =\\
           (1-\rho)(-2\;K^{''}(z)-K^{''}(z)+ (1-z)K^{'''}(z))
    \end{align*}
    Fazendo z = 1 
    \begin{align*}
           &3\;N^{''}(1)\;(K^{'}((z)-1) + 3\;N^{'}(1)\;(K^{''}(1)) + \;(K^{'''}(1)) =(1-\rho)(-3\;K^{''}(1))\\
           &3\;N^{''}(1)\;(1-\lambda E[X]) - 3\;N^{'}(1)\;(\lambda^2 E[X^2]) - \;\lambda^3 E[X^3] =(1-\rho)(3\;\lambda^2 E[X^2])\\
           &3\;N^{''}(1)\;(1-\lambda E[X])=(1-\rho)(3\;\lambda^2 E[X^2]) +3\;N^{'}(1)\;(\lambda^2 E[X^2]) + \;\lambda^3 E[X^3] \\
           &3\;N^{''}(1)\;(1-\lambda E[X])=(1-\rho)(3\;\lambda^2 E[X^2]) +\;\frac{3\;(\lambda^2 E[X^2])((1-\rho)(2\lambda\;E[X])+ \lambda^2\; E[X^2])}{2(1-\lambda E[X])} + \;\lambda^3 E[X^3] \\
    \end{align*}
    \(3\;N^{''}(1)\;(1-\lambda E[X]) = \)
    \begin{align*}
        \frac{6\;(1-\rho)(\lambda^2 E[X^2])(1-\lambda E[X]) +\;+3\;(\lambda^2 E[X^2])((1-\rho)(2\lambda\;E[X])+ \lambda^2\; E[X^2]) + 2(1-\lambda E[X]) (\;\lambda^3 E[X^3] )}{2(1-\lambda E[X])}\\
    \end{align*}
    \(N^{''}(1) = \)
    \begin{align*}
        \frac{6\;(1-\rho)(\lambda^2 E[X^2])(1-\lambda E[X]) +\;+3\;(\lambda^2 E[X^2])((1-\rho)(2\lambda\;E[X])+ \lambda^2\; E[X^2]) + 2(1-\lambda E[X]) (\;\lambda^3 E[X^3] )}{6(1-\lambda E[X])^2}\\
    \end{align*}
    Logo \(E[N^{2}] = N^{''}(1) + N^{'}(1)\) para uma fila M/M/1, pode ser dado por:
    \begin{align*}
        &\frac{12\;(1-\rho)^2\;\rho^2\;+ 6\;\rho^2\; ((1-\rho)\;2\;\rho+ \; 2\;\rho^2) + 12\;(1-\rho)\;\rho^3 + 3(1-\rho)(2\;\rho(1-\rho) + 2\;\rho^2 )\;}{6(1-\rho)^2}\\
        &\frac{6\rho-6\rho^2+12\rho^2-24\rho^3+12\rho^4+12\rho^3+12\rho^3-12\rho^4}{6(1-\rho)^2}\\
        &\frac{6\rho^2 + 6\rho}{6(1-\rho)^2} = \frac{\rho^2 + \rho}{(1-\rho)^2}
    \end{align*}
    Agora podemos calcular \(E[Nq^{2}]\)
    \begin{align*}
        E[Nq^2] &= \frac{\rho^2 + \rho}{(1-\rho)^2} -\frac{2\;\rho}{(1-\rho)} + \rho\\
        &= \frac{\rho^2 + \rho - 2\;\rho + 2\;\rho^2 + \rho - 2\;\rho^2 + \rho^3}{(1-\rho)^2}\\
         &= \frac{\rho^2 + \rho^3}{(1-\rho)^2}
    \end{align*}
    Por fim, calculamos a variância do número de pessoas na fila de espera de uma M/M/1:
    \begin{align*}
        V(Nq) &= \frac{\rho^2 + \rho^3}{(1-\rho)^2} - (\frac{\rho^2}{1-\rho})^2\\
        &=\frac{\rho^2 + \rho^3 - \rho^4}{(1-\rho)^2}
    \end{align*}
\end{itemize}
\begin{itemize}
    \item Tempo médio em uma fila de espera FCFS\\\\
    Acompanhando um freguês tíıpico em uma fila M/G/1 FCFS, podemos observar que o núemro deixado para trás é exatamente o núemro de pessoas que chegou durante o tempo T que o freguês típico passou no sistema. Como o processo de chegada  é Poisson e relacionando as chegadas ao intervalo de tempo em que as chegadas ocorreram, pode-se escrever \(N(z) = T(\lambda - \lambda \;z)\). Fazendo \(s = \lambda - \lambda \;z\) e substituindo \(z = \frac{\lambda-s}{\lambda} \), temos:
    
    \begin{center}
        \(T^{*}(s) = \frac{(1-\rho)sX^{*}(s)}{s-\lambda+\lambda X^{*}(s)}\)
    \end{center}
    Sabemos que T = W + X, como W e X sãp independentes e uma fila FCFS, então  \(T^{*}(s) =  W^{*}(s)\;X^{*}(s)\), logo a tranformada de Laplace da pdf do tempo de espera na fila é dada por:
    \begin{center}
        \(W^{*}(s) = \frac{(1-\rho)s}{s-\lambda+\lambda X^{*}(s)}\)
    \end{center}
    O tempo médio de espera na fila pode ser facilmente obtido derivando \(W^{*}(s)\), uma vez que \(E[W] = - W^{*'}(0)\). Dessa forma, derivando em relação a s temos:
    \begin{align*}
       W^{*}(s)(s-\lambda+\lambda X^{*}(s)) &= (1-\rho)s\\
       W^{*'}(s)(s-\lambda+\lambda X^{*}(s))+ W^{*}(s)(1+\lambda X^{*'}(s)) &= (1-\rho)
    \end{align*}
    Derivando novamente para retirar  a ideterminação:
    \begin{align*}
       W^{*''}(s)(s-\lambda+\lambda X^{*}(s)) + W^{*'}(s)(1+\lambda X^{*'}(s))+ W^{*'}(s)(1+\lambda X^{*'}(s)) + W^{*}(s)(\lambda X^{*''}(s))= 0
    \end{align*}
    Fazendo s = 0 
    \begin{align*}
        2 W^{*'}(0)(1+\lambda X^{*'}(0)) &=  -\lambda X^{*''}(0)\\
        W^{*'}(0) &= \frac{-\lambda X^{*''}(0)}{1+\lambda X^{*'}(0)}\\
        E[W] &= \frac{\lambda E[X^2]}{2(1- \lambda E[X])}
    \end{align*}
    Como \(E[X^2] = 2\;E[X]\;E[Xr]\), podemos escrever \(E[W] = \frac{\lambda\; 2\;E[X]\;E[Xr]}{2(1- \lambda E[X])}\). \\\\
    Para a fila M/M/1, como temos um serviço exponencial, \(E[Xr] = E[X]\), logo \(E[W] = \frac{\rho\;E[X]}{(1- \rho)}\). Para o nosso simulador o tempo médio de serviço é 1 segundo, então \(E[W] = \frac{\rho}{(1- \rho)}\)
\end{itemize}
\begin{itemize}
    \item Variância do tempo de espera em uma fila de espera FCFS\\\\
    Para o cálculo da variância é preciso achar o segundo momento  \(E[W^2] = W^{*''}(0)\). Assim sendo, derivando \(W^{*}(s)\) pela terceira vez:
    \begin{align*}
        W^{*''}(s)(s-\lambda+\lambda X^{*}(s)) + W^{*'}(s)(1+\lambda X^{*'}(s))+ W^{*'}(s)(1+\lambda X^{*'}(s)) + W^{*}(s)(\lambda X^{*''}(s))= 0\\
        W^{*'''}(s)(s-\lambda+\lambda X^{*}(s))+W^{*''}(s)(1+\lambda X^{*'}(s)) + W^{*''}(s)(1+\lambda X^{*'}(s))+W^{*'}(s)(\lambda X^{*''}(s))+\\ W^{*''}(s)(1+\lambda X^{*'}(s))+W^{*'}(s)(\lambda X^{*''}(s)) + W^{*'}(s)(\lambda X^{*''}(s)) + W^{*}(s)(\lambda X^{*'''}(s))= 0
    \end{align*}
    Fazendo s = 0
    \begin{align*}
        &3W^{*''}(0)(1+\lambda X^{*''}(s)) + 3 W^{*'}(0)(\lambda X^{*''}(0)) + \lambda X^{*'''}(0) = 0\\
        &3W^{*''}(0)(1-\lambda X^{*''}(s)) =- 3 W^{*'}(0)(\lambda X^{*''}(0)) - \lambda X^{*'''}(0)\\
        &3W^{*''}(0)(1-\lambda E[X^2]) =3 E[W](\lambda E[X^2]) + \lambda E[X^3]\\
        &W^{*''}(0)= \frac{3\;\lambda E[X^2]\; \lambda E[X^2]}{6\;(1-\lambda E[X^2])^2} + \frac{\lambda E[X^3]}{3\;(1-\lambda E[X^2])}\\
        &W^{*''}(0)= \frac{\;\lambda E[X^2]\; \lambda E[X^2]}{2\;(1-\lambda E[X^2])^2} + \frac{\lambda E[X^3]}{3\;(1-\lambda E[X^2])} \\
        &W^{*''}(0)= 2\; E[W]^2 + \frac{\lambda E[X^3]}{3\;(1-\lambda E[X^2])} = E[W^2]
    \end{align*}
    Logo, para a fila M/M/1 \(E[W^2] = 2\; E[W]^2 + \frac{\lambda E[X^3]}{3\;(1-\rho)}\)
    Finalmente, pemos achar a variância para uma fila M/M/1 FCFS:
    \begin{align*}
        V(W_{FCFS}) &= E[W^2] - E[W]^2\\
        &= 2\; E[W]^2 + \frac{\lambda E[X^3]}{3\;(1-\rho)} - E[W]^2\\
        &= E[W]^2 + \frac{\lambda E[X^3]}{3\;(1-\rho)}\\ &=\frac{\lambda^2\;E[X^2]^2}{4\;(1-\rho)^2} + \frac{\lambda E[X^3]}{3\;(1-\rho)} \\
        &= \frac{3\;\lambda^2\;E[X^2]^2 + 4\;\lambda E[X^3]- 4\rho\lambda E[X^3]}{12\;(1-\rho)^2}
    \end{align*}
    Como o tempo médio de serviço é 1 segundo, \(\lambda = \rho\)
    \begin{align*}
        V(W_{FCFS}) = \frac{3\;\rho^2\;E[X^2]^2 + 4\;\rho E[X^3]- 4\rho^2 E[X^3]}{12\;(1-\rho)^2}
    \end{align*}
    O segundo e o terceiro momentos de uma exponencial, com \(\mu = 1\) podem ser dados por:
    \begin{align*}
       &E[X^2] BOTAR INTEGRAL = \frac{2}{\mu^2 } = 2 \\
       &E[X^3] BOTAR INTEGRAL = \frac{6}{\mu^3 } = 6
    \end{align*}
    Então,
    \begin{align*}
        V(W_{FCFS}) = \frac{-12\;\rho^2+ 24\;\rho }{12\;(1-\rho)^2}
    \end{align*}
\end{itemize}
%\newpage
\section{Tabelas com Resultados e Comentários Pertinentes}
%\newpage
\section{Conclusão}
%\newpage
\section{Anexo - Listagem Documentada do Programa}

\end{document}
